% Options for packages loaded elsewhere
\PassOptionsToPackage{unicode}{hyperref}
\PassOptionsToPackage{hyphens}{url}
%
\documentclass[
]{article}
\usepackage{lmodern}
\usepackage{amssymb,amsmath}
\usepackage{ifxetex,ifluatex}
\ifnum 0\ifxetex 1\fi\ifluatex 1\fi=0 % if pdftex
  \usepackage[T1]{fontenc}
  \usepackage[utf8]{inputenc}
  \usepackage{textcomp} % provide euro and other symbols
\else % if luatex or xetex
  \usepackage{unicode-math}
  \defaultfontfeatures{Scale=MatchLowercase}
  \defaultfontfeatures[\rmfamily]{Ligatures=TeX,Scale=1}
\fi
% Use upquote if available, for straight quotes in verbatim environments
\IfFileExists{upquote.sty}{\usepackage{upquote}}{}
\IfFileExists{microtype.sty}{% use microtype if available
  \usepackage[]{microtype}
  \UseMicrotypeSet[protrusion]{basicmath} % disable protrusion for tt fonts
}{}
\makeatletter
\@ifundefined{KOMAClassName}{% if non-KOMA class
  \IfFileExists{parskip.sty}{%
    \usepackage{parskip}
  }{% else
    \setlength{\parindent}{0pt}
    \setlength{\parskip}{6pt plus 2pt minus 1pt}}
}{% if KOMA class
  \KOMAoptions{parskip=half}}
\makeatother
\usepackage{xcolor}
\IfFileExists{xurl.sty}{\usepackage{xurl}}{} % add URL line breaks if available
\IfFileExists{bookmark.sty}{\usepackage{bookmark}}{\usepackage{hyperref}}
\hypersetup{
  pdftitle={Ecology Workshop Project Proposal: Seasonal Vertical Distribution of Phytoplankton in a Subtropical Dystrophic Lake},
  pdfauthor={Kristy Sullivan},
  hidelinks,
  pdfcreator={LaTeX via pandoc}}
\urlstyle{same} % disable monospaced font for URLs
\usepackage[margin=1in]{geometry}
\usepackage{graphicx,grffile}
\makeatletter
\def\maxwidth{\ifdim\Gin@nat@width>\linewidth\linewidth\else\Gin@nat@width\fi}
\def\maxheight{\ifdim\Gin@nat@height>\textheight\textheight\else\Gin@nat@height\fi}
\makeatother
% Scale images if necessary, so that they will not overflow the page
% margins by default, and it is still possible to overwrite the defaults
% using explicit options in \includegraphics[width, height, ...]{}
\setkeys{Gin}{width=\maxwidth,height=\maxheight,keepaspectratio}
% Set default figure placement to htbp
\makeatletter
\def\fps@figure{htbp}
\makeatother
\setlength{\emergencystretch}{3em} % prevent overfull lines
\providecommand{\tightlist}{%
  \setlength{\itemsep}{0pt}\setlength{\parskip}{0pt}}
\setcounter{secnumdepth}{-\maxdimen} % remove section numbering

\title{Ecology Workshop Project Proposal: Seasonal Vertical Distribution of
Phytoplankton in a Subtropical Dystrophic Lake}
\author{Kristy Sullivan}
\date{1/16/2020}

\begin{document}
\maketitle

\hypertarget{research-statement}{%
\subsubsection{Research Statement}\label{research-statement}}

Physical, chemical, and competative processes can influence the vertical
and seasonal distribution of phytoplankton in freshwater lakes. In
dissolved organic carbon (DOC) rich waters, the photic zone may be
limited to just a few meters below the surface. Due to temperate bias,
current models of phytoplankton succession are unable to predict the
seasonal trends observed in many subtropical and tropical monomictic
lakes. In subtropical dystrophic Lake Annie (Highlands County, FL, USA),
dissolved organic carbon quality, quantity, and resistance to mixing may
be strong drivers of phytoplankton assemblages spatially and temporally.

\hypertarget{objectives}{%
\subsubsection{Objectives}\label{objectives}}

My goals in this workshop are to develop spatiotemporal species
distribution models of dominant phytoplankton species in response to
environmental drivers (i.e.~DOC, temperature, nutrients). I hope to use
the results of this project as a chapter of my masters' thesis.

\hypertarget{hypotheses}{%
\subsubsection{Hypotheses}\label{hypotheses}}

\begin{enumerate}
\def\labelenumi{\arabic{enumi}.}
\tightlist
\item
  Temporal Patterns: Species diversity and richness will be greatest in
  dark years due to increased resource availability (high DOC).
  Mixotrophic to autotrophic phytoplankton ratios will also increase
  with greater concentrations of DOC.
\item
  Vertical Patterns: Diversity and richness will be greatest in the
  epilimnion due to greatest light availability.
\end{enumerate}

\hypertarget{datasets}{%
\subsubsection{Datasets}\label{datasets}}

The datasets to be used in this project include a personal dataset of
phytoplankton absolute abundance, DOC quality and quantity, bacterial
abundance, dissolved oxygen, and temperature measured quartly (November
2018 - January 2020) from five depths (surface, 0m; chl-maximum, 2-4m;
epilimnion, 6-9m; thermocline, 7-12m; and hypolimnion, 15m) from the
central buoy of Lake Annie in Highlands County, FL. A second 14 year
dataset of environmental parameters and phytoplankton net tows taken
monthly from the center buoy at a depth of 10 meters (2006-2019) will be
accessed with permission from Archbold Biological Station. These
datasets will be publically available from the Archbold Biological
Station data repository:
\url{https://www.archbold-station.org/html/datapub/data/data.html}

\hypertarget{statistical-analyses}{%
\subsubsection{Statistical analyses}\label{statistical-analyses}}

A preliminary NMDS analysis will be used to determine the common
predictors of the observed phytoplankton distributions across time. From
these data, I will be able to identify which species commonly occur
together, and what environmental drivers influence their abundances. I
will use these data to create a model which will predict the seasonal
species distribution of phytoplankton species in Lake Annie given input
values of the most influential drivers (i.e.~Schmidt stability, DOC
concentration/quality, and nutrients). I hope to create a model similar
to the well established PEG model of temperate dimictic lakes (Sommer et
al., 1986).

\hypertarget{preliminary-results}{%
\subsubsection{Preliminary Results}\label{preliminary-results}}

Figure 1 (below) shows the relative abundance of the eight phyla of
phytoplankton found in Lake Annie from monthly 10 m vertical net tows.
There is a high relative abundance of \emph{Bacillariophyta} (yellow)
from years 2006-2010, while they are essentially absent years 2010-2013
at which point \emph{Cyanophyta} and \emph{Ochrophyta} become most
dominant. This could be due to a decreased bioavailability of silica for
the diatoms or perhaps a shift in nutrients. Most likely, there are
multiple drivers with additive effects causing the observed abrupt shift
in phyla relative abundance.

\begin{figure}
\centering
\includegraphics{C:/Users/ksull/Pictures/Kristy_Sullivan_LA_2006_2012.jpg}
\caption{alt text}
\end{figure}

\end{document}
